%% ----------------------------------------------------------------
%% Introduction.tex
%% ---------------------------------------------------------------- 
\chapter{Introduction} \label{Chapter:Introduction}

The promising expansion of the Internet of Things (IoT) has drawn research interests on new design paradigms for deploying tens of billions of electronic devices over a wide geographical range and probably in hard-to-reach places~\cite{hahm2016operating, mainetti2011evolution}. Such a scenario generates considerations on how to enable the devices in networks to operate independently and effectively and how to construct a long-life, maintenance-free, environmentally friendly, and low-cost IoT.

\section{Energy Harvesting Computing}
% consider adding an abbreviation of energy harvesting sensor nodes?

\subsection{Energy Harvesting and Energy-Neutral Computing}

One of the most significant concerns in deploying IoT devices is how to power numerous low-power devices (tens of billions expected~\cite{hahm2016operating, adegbija2017microprocessor, shi2016edge}). Traditional wired electricity limits flexibility of deployment and involve expensive wiring costs~\cite{rabaey2000picoradio}. Primary batteries (i.e. non-rechargeable batteries) cannot support such a large number of devices, since the widespread use of primary batteries will cause heavy battery replacement work due to the limited battery lifetime as well as pollution issues. Therefore, it is necessary to find an alternative powering solution.

A potential powering solution is to employ energy harvesting techniques. Energy harvesting denotes scavenging energy from environmental sources (e.g. solar irradiation, wind flow, radio frequency (RF) signals, and kinetic energy)~\cite{mitcheson2008energy}. Devices powered by energy harvesters get rid of wires and surpass the lifetime limit of primary batteries, enabling a scalable IoT. However, the power generated by energy harvesters in real-world deployment is variable, uncontrollable, and intermittent, and hence, hinders forward execution if directly connected (without energy buffering) to loads~\cite{chalasani2008survey}.
% Scalability: the ability of the network/sth else to operate efficiently when the number of nodes is dramatically increased. Mobility: movable.

% Energy storage is usually allocated in combination with energy harvesters in the power supply system of autonomous devices in order to balance the temporal variability of energy sources and provide a stable power supply~\cite{sudevalayam2011energy}. 
Initially, large energy storage, in forms of rechargeable batteries or supercapacitors, is allocated with energy harvesters to buffer the temporal variations of energy sources and provide reliable power supply. Energy-neutral (EN) computing was derived from such a method of power buffering~\cite{sudevalayam2011energy}. EN computing aims to sustain systems over a long period (e.g. a few days~\cite{kansal2007power} or a year~\cite{buchli2014dynamic}) by adapting system run-time schedules (e.g. duty cycles~\cite{kansal2007power, buchli2014dynamic, le2012power} or task schedules~\cite{caruso2018dynamic, wagemann2018operating}) according to the available energy amount. 

Rechargeable batteries (also known as secondary batteries) and supercapacitors are two main choices of energy storage in EN computing. Rechargeable batteries are historically used as energy storage in energy harvesting embedded systems because of their high energy density~\cite{akhtar2015energy} and stable discharging profile~\cite{sudevalayam2011energy}. However, due to electrolyte deterioration, the limited charge-discharge cycles of rechargeable batteries constrain the operating lifetime, causing heavy battery replacement work as well as environmental issues as primary batteries~\cite{rakhmatov2002battery}. To alleviate the problems of batteries, supercapacitors are then explored in research. Although the energy density of supercapacitors is several orders of magnitude lower than the energy density of batteries~\cite{merrett2012supercapacitor}, supercapacitors outperform rechargeable batteries in terms of lifetime (e.g. 10-20 years for supercapacitors and 3-5 years for rechargeable batteries~\cite{simjee2008efficient}). However, to achieve a comparable energy capacity to batteries, supercapacitors should be designed to tens of farads or one hundred farads~\cite{jiang2005perpetual, simjee2006everlast}; supercapacitors in such a scale occupy large volume in small IoT devices. 
% \cite{torah2008self} uses 47mF supercapacitor, consider this later, jiang2005perpetual 2 x 22F, simjee2006everlast 100F

\subsection{Storage-Less Energy Harvesting Computing}

To circumvent the maintenance, pollution, and volume problems in rechargeable batteries and supercapacitors, a recent research trend of energy harvesting sensor nodes switches towards eliminating the demand for energy storage and adopting only a minimum amount of energy storage (denoted as storage-less, where energy storage is only enough for ensuring the most energy-expensive atomic\footnote{In this context, an operation is atomic if it should be completed in one consecutive period without power interrupts; otherwise, if interrupted, it should be re-executed from the beginning. Example atomic operations in IoT devices can be peripheral operations and nonvolatile memory read/write operations.} operation of systems), typically in the form of a capacitor. However, given a negligible amount of energy storage, it becomes a major research concern that how to guarantee forward execution and functionality of energy harvesting sensor nodes .

Avoiding large energy storage makes IoT devices long-life, maintenance-free, environmentally friendly, and compact (small in dimensions). However, in energy harvesting sensor nodes, replacing batteries and supercapacitors with small capacitors considerably limits the buffering capacity for varying harvested power, and hence, the variable power is directly given to the load. This violates the demand for stable power supply in conventional computing architecture and causes program execution to be frequently stranded in the same portion of code due to recurring power outages. 

% When the number of deployed sensor nodes in IoT increases in orders of magnitude, network data traffic and energy consumption in communication will increase accordingly and become a serious issue.
% However, improving local data processing ability of sensor nodes is crucial to reducing energy consumed in communication and data traffic in sensor networks~\cite{akyildiz2002wireless}.
% include not only basic sensing and communicating functions, but also stronger local data processing and controlling ability. 

However, ensuring and improving local processing ability of sensor nodes is crucial for the following reasons. First, sensor nodes should be able to process sensing data on-site and transmit only the useful information, such that the network traffic volume and energy consumption is reduced~\cite{shi2016edge}. Second, advanced communications techniques (such as scheduling, routing, coding, decoding) require local computing ability to ensure timeliness and efficiency in networking~\cite{akyildiz2002wireless}. Third, IoT objects are also expected to be able to trigger actions in reaction to the physical reality by either receiving commands from other nodes and servers or making a decision based on locally acquired data~\cite{miorandi2012internet}. 

In order to adapt conventional computing architecture to energy harvesting power supply without large energy storage, many methodologies for storage-less energy harvesting computing have emerged in the past few years, with considerations on forward execution, computation correctness, and improving execution speed. Recently, there are two research topics in storage-less energy harvesting computing: intermittent computing (IC) and power-neutral (PN) computing. A summary of this research trend in energy harvesting computing is shown in~\fref{Figure:paradigm}.
% Briefly speaking, intermittent computing maintains system computing state after power outages, and PN computing aims to make more progress from available power by matching system power consumption with power harvested instantaneously.

\begin{figure}[!htb]
  \centering
  \includegraphics[width=12cm]{figure/intro/paradigm}
  \caption{Research trend in energy harvesting computing: towards minimising energy storage.}
  \label{Figure:paradigm}
\end{figure}

Intermittent computing aims to maintain system volatile state after power outages with low time and energy overheads, so as to ensure forward execution and computation correctness of applications~\cite{ransford2012mementos}. Approaches in intermittent computing fundamentally diverge due to different design goals and can be classified into five categories: checkpointing, reactive, harvest-store-use, task-based, and non-volatile processors. Each of these approaches has one or a few advantages on reducing the number of system snapshots during a power cycle, reducing the size of system snapshots, minimising hardware dependency, etc~\cite{sliper2018enabling}. 

PN computing aims to make more progress from available power by matching system power consumption with power harvested instantaneously. PN computing dynamically adapts system performance, typically by scaling CPU clock frequency, to match the instantaneous system power consumption with the instantaneous harvested power~\cite{balsamo2016graceful, fletcher2017power}. By such performance adaptation, PN systems immediately consume excessive harvested power, prolong system active time when the harvested power drops, and reduce the frequency of system state saving and restoring operations.

\subsection{Application Suitability of Storage-less Energy Harvesting Computing}

An inherent limitation of storage-less energy harvesting computing is that the systems can only execute when there is available power, as opposed to EN computing where the EN systems can still executes with buffered energy if ambient power is not available. This limitation thereby requires that \textit{application operation periods and power availability should be compatible in time}. While there are various needs of application operation periods for various applications scenarios, the power availability is constrained and determined by the availability of the target energy source in the deployed environment. The applications of storage-less energy harvesting computing should be adapted or selected to suit the power availability. Under this consideration, there are two typical categories of application scenarios according to recent publications. 

\begin{itemize}

  \item \textbf{Category 1: Applications with flexible time requirements.}
  
  Applications with flexible requirements on operating periods tolerate the intermittency of energy harvesting sources. In such applications, energy-harvesting devices are allowed to wait for power-available periods to execute.
  % Example energy sources could be outdoor solar energy, indoor radio-frequency (RF) energy, and human body thermal energy. Given such energy conditions, the application can process information irrelevant to the energy sources. As such energy sources are probably available in scattered and irregular periods, the application should wait for energy-available periods to activate execution. Delay-insensitive applications tolerate flexible operation periods, and hence, are suitable for such energy conditions.

  \textbf{Use case 1: Kitchen event detection} 
  
  The application aims to capture kitchen events (e.g. dishwasher working, fan on, refrigerator cooling) to record equipment usage. As such events usually last for from tens of seconds to a few hours, the device does not need to operate immediately after the event occurs or disappears. An implementation of this application is shown by Maeng et al.~\cite{maeng2019supporting}. The device iterates the following tasks in turn during power-available periods: sampling acoustic information from microphone input, classifying kitchen events with a pre-trained model, and transmitting the results in Bluetooth Low-Energy (BLE) packets to an always-on server. The device harvests RF energy from a dedicated RFID reader, and the packets are transmitted every a few seconds as reported. A specification for this application is to complete program iterations as frequently as possible so as to improve the accuracy of event records. 

  \hfill

  \textbf{Use case 2: Temperature monitor for air conditioning}

  The application aims to monitor indoor temperature for air conditioning. As the temperature does not usually change over a few minutes, the temperature monitor does not need to wake up frequently or periodically. An implementation of this application is shown by Colin et al.~\cite{colin2018reconfigurable}. During power-available periods, the device samples temperature by an external analog sensor. If the temperature is detected to be out of a pre-defined range, the device send a BLE packet to alarm the server. The device is also powered by a dedicated RFID reader. Similar to use case 1, the device is expected to maximise sampling frequency in order to capture out-of-range temperature as soon as possible.

  \item \textbf{Category 2: Application activity in correlation with available power.}
  
  In such applications, the application operations correlates with power-available periods. This correlation is typically linked by an event with harvestable power. When the event occurs, the device is activated by harvesting the power of the event at the same time to start operating. Therefore, the application operation periods and the power availability are inherently simultaneous in such applications. 

  \textbf{Use case 3: Bicycle trip counter}

  The bicycle trip counter aims to calculate cycling speed and traveled distance. The wheel rotation brings energy for the device to sense the cycling speed; the device does not need to operate without cycle movement. The trip counter is designed as a nail-sized chip installed on the frame of a bicycle, with a magnet on the wheel for gaining energy~\cite{bing2018energy}. Every wheel rotation activates the trip counter to calculate the current speed and log traveled distance. After collecting enough energy over a few cycling rounds, the trip counter transmits the logged information. This application is also expected to complete computing tasks faster and report results as frequently as possible. 

  \textbf{Use case 4: Power meter}

  The power meter measures the power flow of a main load wire. The AC power in the wire can be harvested by a coil to activate the power meter. A design is shown in Monjolo~\cite{debruin2013monjolo}, where the power meter transmits a plain packet to a server once it collects a preset amount of energy. The server then calculates the elapsed time between the recent two packets to estimate the main load power. 

\end{itemize}
  
As shown in the above use cases, a common application specification of storage-less energy harvesting computing is to do as much 'work' as possible under the same energy conditions, e.g. completing program iterations as frequently as possible, because the energy cannot be saved for later use. Other 'work' could include improving sensing accuracy or processing offloaded tasks received from other devices. To summarise the application suitability of storage-less energy harvesting computing, a diagram is shown in~\fref{Figure:appsuit}. An example unsuitable application can be a periodic sensing task without periodically available power or the period of the energy source does not match the sensing period (left bottom circle in~\fref{Figure:appsuit}).

\begin{figure}[!htb]
  \centering
  \includegraphics[width=12cm]{figure/intro/appsuit2}
  \caption{Application Suitability of Storage-less Energy Harvesting Computing.}
  \label{Figure:appsuit}
\end{figure}

% \subsection{General Architecture of Energy Harvesting Sensor Nodes}

% According to recent publications of energy harvesting computing~\cite{naderiparizi2015wispcam, gomez2016dynamic, sun2017maximum, wang2016storage, balsamo2016graceful, gomez2017wearable, buchli2014dynamic}, a typical architecture of energy harvesting sensor nodes is shown in~\fref{Figure:architecture}. There are five major components: energy harvester, power management unit (PMU), energy storage, voltage converter, and load. 

% \begin{figure}[!htb]
%   \centering
%   \includegraphics[width=13cm]{figure/intro/architecture}
%   \caption{Typical architecture of energy harvesting sensor nodes.}
%   \label{Figure:architecture}
% \end{figure}

% The energy harvester is the only power source in such nodes. The harvested power is then transferred to the energy storage, in some cases through a PMU. The function of PMU is to efficiently convert harvested voltage and current to an operational state for the energy storage and the load, e.g. to maximise the harvested power from the energy harvester by controlling operating voltage in systems powered by photovoltaic (PV) panels~\cite{wang2016storage, sun2017maximum, patel2008maximum}. The energy storage acts as a buffer to smooth out the inevitable and frequent temporal disparity between power harvested and power consumed, which is designed as either a rechargeable battery with a large capacity and a short lifetime, or a supercapacitor with a small capacity and a long lifetime, or in a hybrid form~\cite{xie2013charge}. With recent advances in IC, this energy storage can be minimised to a decoupling capacitor or even eliminated (no additional storage). The energy in the storage is then consumed by the load. Usually, when the terminal voltage of the energy storage does not match the operating voltage of the load, voltage conversion circuits (such as a voltage regulator, a DC-DC converter, or a low-dropout regulator) are required to provide a voltage output within the load operating range~\cite{naderiparizi2015wispcam, gomez2016dynamic, wu2012efficient}. The load, in a typical IoT application, is normally a microcontroller (MCU) with one or a few sensors and a radio. 

\section{Research Justification}

Self-powered devices solely rely on energy harvesters to power themselves. Energy harvesting power supply varies over time due to uncontrollable environmental conditions. Due to the lifespan, pollution, dimension problems of batteries and supercapacitors, there is a demand for adapting computing architecture to energy harvesting supply with small storage, e.g. a capacitor. Various approaches in intermittent computing and PN computing emerge to enable and improve energy harvesting computing in the scenario of minimised storage in order to reduce dimensions and cost as well as to increase lifespan of IoT devices. However, a larger storage than the minimum one can reduce the frequency of power outages and increase flexibility in power management, which may benefit the application throughput. Hence, there is a trade-off in sizing energy storage. Apart from energy storage, the size of energy harvesters should also be considered based on a set of factors, such as system power consumption, performance requirements, energy utilization, device dimensions, and cost. A gap of study exists in how to determine the size of energy storage and energy harvesters. The research gap is further described as listed below:
\begin{itemize}
  \item [1.] A variety of methods in intermittent computing have been proposed to overcome power outages during execution, ensuring forward execution~\cite{ransford2012mementos}, memory consistency~\cite{lucia2015simpler}, peripheral re-configuration~\cite{liu2019130}, etc. Such approaches usually adopt only a minimum amount of energy storage which only allow systems to complete state saving and restoring operations before and after power outages. However, provisioning a certain amount of storage reduces the overhead of intermittent operations. This effect is especially noticeable in reactive intermittent systems (Section \ref{Section:reactiveic}), where the energy and time overheads of saving and restoring operations are proportional to the frequency of power outages. On the other hand, increasing energy storage also increases the system leakage power and affects the reactivity (the energy and the charging time to activate devices) of intermittently-powered systems~\cite{colin2018reconfigurable, wu2018extensible}. The trade-off of scaling energy storage in intermittent computing systems should be fully studied. 
  \item [2.] The size of energy harvesters significantly determines the amount of harvested power. Undersized energy harvesters constrain available power, affecting the maximum performance of devices; oversized energy harvesters provides redundant energy, which cannot be consumed on useful work and is wasted in circuitry or never harvested, affecting the utilization of energy harvesters and unnecessarily increasing device costs. Besides, the size of energy harvesters contributes to a large part in device dimensions, which may violate the size constraint of autonomous devices~\cite{buchli2014dynamic}. Therefore, there is a trade-off between under-provisioning and over-provisioning energy harvesters in energy harvesting devices, and this trade-off should be thoroughly studied. 
  \item [3.] Current power management techniques in energy harvesting computing rely on the feedback of the dynamic detection of available energy in energy storage, whether in EN systems~\cite{kansal2007power, wagemann2018operating} or in PN systems~\cite{balsamo2016graceful, fletcher2017power}. Scaling the capacity of energy storage may change the responsiveness of such energy detection, and consequently, have an impact on performance adaptation and system application throughput. This impact exhibits more significant in PN computing than in EN computing as the allocated energy storage in PN systems is minimised and energy detection is more acute. 
\end{itemize}

\section{Research Questions}

Based on the research justification, three research questions are listed below:

% Questions
% 1. Does the energy storage size affect the performance of intermittent computing? how/why? 
% 2. How can developers decide the size of energy storage? (This question may be challenged by “what about other energy storage architecture?”)
% 3. After addressing the efficiency of computing tasks, how can the devices perform atomic tasks, e.g. peripheral operations. (After this question, my previous questions may be challenged by “why doesn’t that encompass atomic tasks?”)


\begin{itemize}

  % \item[1.] What is the effect of energy storage capacity and energy harvester size on the behaviours of intermittent computing systems?
  
  % Adding energy storage to intermittent computing systems may reduce state saving and restoring overheads and tolerate larger atomic operations, but may also increase leakage power, device dimensions, and undermine system reactivity. Energy harvester sizing also affects multiple outcomes, such as system performance requirement, device dimensions, and costs. System behaviours include application throughput, active time, reactivity, frequency and overheads of state saving and restoring operations, and energy utilization. 

  % \item[2.] How can designers size energy harvesters and energy storage in IC systems to maximise application throughput while meeting requirements for device dimensions and costs?
  
  % Based on the analysis of sizing effect on system behaviours, this question focuses on practical concerns when deploying IoT sensor nodes. A sizing tool should be provided, where the user should import the energy source conditions at the deploying location. This tool should provide a spectrum of energy harvester and storage sizes with corresponding outcomes of each sizing choice.

  \item[1.] What is the effect of energy storage capacity on the forward progress of intermittent-powered systems? 
  
  The amount of forward progress directly determines application performance, e.g. program iteration rate or task completion time. 
  Most intermittent-powered systems minimise energy storage capacity so as to minimise device dimensions and interruption periods~\cite{7442814, 10.1145/2700249, 10.1145/2809695.2809707, 10.1145/3281300, 222579}. 
  However, increasing energy storage capacity can also benefit forward progress. 
  Our goal is to explore and quantify the relationship between energy storage capacity and forward progress and to analyse the reasons behind this relationship. 

  \item[2.] How can developers size energy storage in intermittently-powered systems to trade off forward progress against device dimensions and interruption periods?
  
  Based on the knowledge of sizing effect on forward progress, this question encompass more design factors in sizing energy storage when deploying intermittently-powered systems. 
  A simulation tool of intermittently-powered systems should be provided, where users can define energy source conditions and energy harvester sizes. 
  The tool should be able to output the physical size of energy storage and interruption periods in certain metrics. 
  An appropriate size of energy storage can be suggested by this tool with a cost function to trade off multiple factors. 

  \item[3.] How can intermittent-powered systems exploit energy storage so as to perform atomic operations and cope with capacitor degradation? 
  
  Atomic operations in intermittent-powered systems denote operations that should be completed in one continuous period. 
  If an atomic operation is interrupted by a power failure, it should be re-executed rather than resumed. 
  Current approaches handle atomic operations by accumulating energy to a predefined voltage threshold; after reaching that threshold, the system starts execution and re-executes the atomic section if the power fails. 
  However, this threshold is typically set high enough for all atomic operations in a program, so the systems usually wait for a long charging period which wastes energy and impedes response time. 
  Also, when capacitor degrades, the predefined threshold may not provide enough energy for the same operations, so that the system cannot make forward progress. 
  A new method of utilizing energy storage to solve these two problems is necessary. 
  
\end{itemize}

\section{Contributions}

% \footnote{With the limits on portability of different IC approaches, Question 1 and 2 are studied based on reactive IC methodology}

The contributions done to the address the research questions in this thesis are:
\begin{itemize}
  \item[1.] Exploration and analysis of the energy storage sizing effect on reactive intermittent computing system, where we quantify and explain the relationship between energy storage capacity and forward progress.
  \item[2.] A method and a simulation tool for sizing energy storage in deploying intermittent-powered systems, where forward progress, dimensions, and interruption periods are traded off in a cost function.
  \item[3.] An online task profiling and voltage threshold adaptation approach for efficiently performing atomic operations and coping with capacitor degradation. 
  % \item[4.] A configurable model of reactive IC systems, including the interrelationship among energy harvesting supply, energy storage, and a processing load.
\end{itemize}

% \section{Thesis Organization}

% The remaining part of this report is structured as followed. Chapter~\ref{Chapter:Review} reviews the research progress in energy harvesting computing, such as EN computing, intermittent computing, and PN computing, as well as an introduction of various energy harvesters and energy storage used in energy harvesting computing. Chapter~\ref{Chapter:Work1} analyses the effect of scaling energy storage and energy harvesters in intermittent computing systems, and provides a method of designing these two elements for IoT deployment. Chapter~\ref{Chapter:Work2} presents a study on how the energy storage capacity affects the application throughput in PN systems. Chapter~\ref{Chapter:Conclusion} concludes this thesis and plans future research. 