\section{Related Work} \label{sec:c5_review}

% Outline
% Review the profiling methods/concepts in prior work.
% Peripheral papers
%   1. How they manage to do peripheral operations?
%   2. Are they supposed to fail in our scenarios?
%       - DEBS fails (if no margin) or inefficient (if large margin)
%       - Samoyed is not likely to fail because of large energy storage and subdivision, but inefficient charging because of large energy budget
%       - Samoyed and RESTOP have overheads: saving state, and restoring state if interrupted. RESTOP: redo-logging, Samoyed: undo-logging. 
%   Threshold setting: low - risky, efficient vs high - safe, inefficient


% \subsection{Intermittent Peripheral Operations}

% Intro
% While prior work in intermittent computing have been able to efficiently maintain forward progress on computational workloads, intermittent peripheral operations are not widely studied. 
% This is typically achieved by carefully saving volatile computing state (e.g. data in CPU registers, SRAM) into NVM before a power failure, and restoring it on power recovery. 
% Some of the published approaches intrinsically guarantee the atomicity and termination of peripheral operations, while others focus on computational workloads without support for peripherals. 
Several existing designs have been able to handle atomic peripheral operations in IPSs, where energy profiling of workloads is an inherent part of their methodologies.

% DEBS
\debs{}~\cite{gomez2016dynamic} experimentally profiles the energy consumption of each task at design time, and designates a threshold to each task individually.
After completing an operation, \debs{} enters a low-power mode (LPM) and waits for energy to be replenished to the next threshold. 

% Samoyed, Alpaca
Samoyed~\cite{maeng2019supporting} utilises a custom design-time \textit{energy profiler} to identify an energy storage size that suffices to run an adequate number (hundreds, as suggested) of peripheral operations in one active cycle. 
At runtime, Samoyed starts execution when energy is refilled to a certain threshold, and keeps executing until energy is depleted. 
% Samoyed is a proactive approach where the program knows the boundaries of atomic operations, such that it only outputs valid data on the completion of an atomic operation and disables checkpoints during execution. 
Samoyed differs from proacitve intermittent computing approaches, e.g. Alpaca~\cite{maeng2017alpaca}, mainly on handling computational workloads where it reactively checkpoints when the buffered energy is below a threshold, and supports user-customised subdivision of peripheral operations when the operation cannot complete in one active cycle. 
% Samoyed is unlikely to fail (non-termination) because it assumes a very large capacitor and supports, but it can be inefficient due to the poor charging efficiency (or you need an EMU, like bq25504, which also consumes energy).

% RESTOP (similar to re-execution)
RESTOP~\cite{rodriguez2018restop} provides programmer-configurable rules that track the instructions issued to peripherals through serial interfaces in a history table.
On power recovery, RESTOP re-issues instructions saved in the history table and then resumes the interrupted operation. 
At design time, RESTOP needs to profile the worst-case energy consumption for restoring peripheral state to identify the minimum (most-efficient) restore threshold. 

% Review Adaptive?
% Online profiling but inefficient for long, low-power operations

% Review other papers about or support peripherals
%   Karma? Sytare?


As reviewed above, prior work profiles the energy consumption of atomic peripheral operations at design time to determine a voltage threshold or a capacitor size that avoids non-termination. 
However, this does not actually guarantee the completion of every atomic operation because energy consumption can change with any runtime conditions different to the profiling setup (demonstrated in \sref{subsec:dynamic_energy_consumption}). 
% Hence, previous designs typically achieve atomicity by carefully managing the system state, such that the atomic operation does not output valid data until completion, and, if a power failure happens during the operation, the system state rolls back to the start of the operation upon power recovery (rather than checkpoint and resume). 
Hence, to tolerate dynamic variations, previous designs should usually leave an inefficiently large margin when allocating energy budgets. 
If this large margin is not given, they can cause either non-termination or high overheads of tracking and restoring state, where \debs{} fails, Samoyed undo-logs the NVM data, and RESTOP re-issues peripheral instructions. 

% Comments: Justify Samoyed and DEBS as comparisons in this paper? 

