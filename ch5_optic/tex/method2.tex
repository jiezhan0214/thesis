\section{\nn{} Runtime Energy Adaptation} \label{sec:method2}

% (may goes to Method 2) 
% \item Ability to monitor the supply voltage and signal the MCU to wake up or sleep when a high or low threshold is hit. 
% Intermittent systems are also equipped with voltage monitoring function to determine when to wake up or sleep, which may be achieved by a voltage comparator~\cite{kang2018homerun, balsamo2016hibernus++}, an energy management unit~\cite{gomez2016dynamic, maeng2019supporting}, or a constant ADC polling. 

Support three types of adaptation.

With a target end voltage $\symb{V}{end}$ below which the system fails, the voltage threshold $\symb{V}{th}$ that the system should set to ensure a task's completion is
\begin{equation}
    \symb{V}{th} = \symb{V}{end} + \Delta \symb{V}{task}
\end{equation}

\subsection{Profiling Strategy}

When should we take a profiling measurement?

Goal: Reduce unnecessary/redundant measurements and take necessary measurements.

\subsection{Learning Algorithm}

How do we use the measurements to update the voltage threshold?

Enable voltage threshold adaptation against runtime variation of energy consumption due to unforeseeable operating conditions, and also enable linear adaptation to function knobs that are known to the system. 

\begin{itemize}
    \item Without function knobs: use the latest profiled threshold.
    \item With function knobs: do a linear regression based on a number of recent measurements. 
\end{itemize}

\subsection{Discussion}

The ability to fall back differentiates it from a "simply-incrementing-threshold" DEBS.
