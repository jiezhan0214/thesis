\section{Introduction}

% *** Background of energy harvesting and intermittent systems ***

%Energy harvesting has become a promising power solution for the Internet of Things, liberating wireless sensors from batteries and the power grid~\cite{sliper2020energy-driven}. 
%Batteryless devices harvest ambient energy, such as light, radio-frequency, and mechanical vibration~\cite{sravanthi2008survey, shaikh2016energy}, which is then buffered in a capacitor. 
%As the harvested power is typically insufficient for continuous operation, such devices operate in an intermittent way -- when a certain amount of energy is collected, the processor wakes up, executes program until the amount of energy falls below a threshold, where it sleeps or dies, and waits for the next active cycle\footnote{An active cycle denotes a continuous period that the intermittent system actively executes workloads, i.e. from when it wakes up till it dies or sleeps. }. 
%Prior work in \textit{intermittent systems} has developed sophisticated methods to preserve forward progress across frequent power interruptions by carefully \textit{checkpointing} the volatile computing state in CPU registers and volatile memory into non-volatile memory (NVM), and restoring the state after power interruptions~\cite{umesh2021survey}. 


% *** Previous work on intermittent peripheral operations ***

Apart from computing, embedded sensor systems need to utilize peripherals, such as sensors, computational accelerators, and radios, which typically require \textit{atomicity}~\cite{berthou2020formal}.
In the context of IPSs, an atomic operation should not be checkpointed during execution; if interrupted by power failures, it should restart rather than checkpoint and resume.
% A peripheral operation is considered atomic because it is infeasible to completely read, save, and restore the intermediate internal state of peripherals, and even if possible, could produce unwanted results (e.g. violating timeliness). 
A peripheral operation is considered atomic because it is usually problematic to checkpoint and restore the operation later, even if the intermediate peripheral state is also checkpointed.
For example, checkpointing during a sensor reading and resuming it later can cause incorrect results or an infinite wait as the initialization is lost, and violate timeliness as the sensor does not render the latest and consecutive results~\cite{maeng2019supporting}. 
% disable checkpoints during execution
Prior works on intermittent peripheral operations either customize a design-time calibrated energy budget for each peripheral operation individually~\cite{gomez2016dynamic}, or allocate a universal and large energy budget that ensure the most energy-hungry operation can finish in one active cycle~\cite{maeng2019supporting}.




% *** Offline profiling and fixed threshold is impractical due to variability ***

However, in this paper we argue that manually profiling each peripheral operation and customizing energy thresholds is impractical due to variability in intermittent systems, where we have considered the variability in the data amount to process, peripheral configurations, devices, and energy buffering capacitance (detailed in Section~\ref{subsec:dynamic_energy_consumption}). 
A fixed threshold can be violated if any of the above cases happen, and lead to non-termination\footnote{Non-termination happens when the pre-defined energy budget is less than how much the operation consumes and the supply is not strong enough to fill the energy gap. It is one of the main causes for failures in intermittent systems. }.
In practical deployment, considering the complexity and labour effort, it is unrealistic to profile every atomic operation for every device under every runtime scenario at design time and customize the energy budgets accordingly. 



% *** An optimized threshold improves efficiency ***

On the other hand, using only one high voltage threshold, though probably avoids non-termination, can affect system energy efficiency. 
% Microcontrollers and peripheral devices typically draw more current at a higher supply voltage. 
Intermittent systems typically minimize operating voltage in order to lower quiescent power consumption from power conversion loss and system leakage~\cite{gomez2016dynamic}. 
Also, a high operating voltage can decrease the output current of energy harvesters, making it harder to charge up the buffering capacitor~\cite{pan2017maximize}.
Hence, setting a high wake-up voltage threshold results in a superlinear long charging time, which therefore slows down the system execution or even leave the system in an infinite wait at low input power.
% \todo{Illustrate or demonstrate this?}


% *** What we do to address it ***

To address the above issue, we propose \nn{}\footnote{\nn{}: \underline{O}nline Energy \underline{P}rofiling and \underline{T}hreshold Adaptation for \underline{I}ntermittent \underline{C}omputing Systems. }, a methodology that profiles energy consumption of operations at runtime and dynamically adapts energy thresholds based on newly profiled consumption and user-defined parameters. 
A naive approach of runtime energy profiling can be disconnecting the power supply during profiling and taking two readings of supply voltage before and after an operation~\cite{zhan2020adaptive}, but this can waste the harvested energy during the operation. 
In contrast, \nn{} profiles the maximum drop of supply voltage that an operation can cause while the energy harvesting supply is connected. 
The profiling strategy is to measure the input current in the charging cycle so as to calculate the maximum drop of supply voltage in the discharging cycle. 
The runtime profiled energy budget can closely match the latest energy consumption of an atomic operation. 
Based on the profiling results, \nn{} dynamically adapts the threshold for each atomic operation, with an option of scaling threshold by user-defined parameters or peripheral configurations.
Therefore, \nn{} avoids non-termination and achieves high energy efficiency, improving the workload throughput.

The main contributions of this article are as follows:

\begin{enumerate}
    \item Design exploration (Section~\ref{sec:design_exploration})
    \item Methodology (Section~\ref{sec:method1} and Section~\ref{sec:method2})
    \item Implementation (Section~\ref{sec:implementation})
    \item Experimental evaluation (Section~\ref{sec:experiment})
\end{enumerate}