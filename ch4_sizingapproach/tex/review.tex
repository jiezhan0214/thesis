\section{Related Work} \label{sec:c4_review}

To explore forward progress of IPSs, simulation tools need to represent transient operation (timescales of \SI{}{\micro\second}-\SI{}{\milli\second}) as well as long-term overall performance (from days up to years). 
A number of models have been proposed for exploring system designs and parameters in IPSs.

Su \textit{et al.}~\cite{Su:2019:TFR:3340300.3320270} modelled a dual-channel solar-powered nonvolatile sensor node, and Jackson \textit{et al.}~\cite{Jackson:2019:COC:3302506.3310400} provided a model to explore battery usage in IPSs. 
Both were configured for long-term simulations and large energy storage (from \SI{}{\milli\farad}-scale supercapacitors to batteries), thus cannot respond to frequent power interruptions and accurately estimate forward progress when using minimized energy storage (e.g. \SI{4.7}{\micro\farad}~\cite{10.1145/3281300}).

In contrast, a set of fine-grained models have been proposed to accurately simulate the frequent micro-operations in IPSs. 
NVPsim~\cite{7428003} is a gem5-based simulator for nonvolatile processors.
Fused~\cite{sliper2020fused} is a closed-loop simulator which allows interaction between power consumption, power supply, and forward progress. 
EH model~\cite{8574572} can compare a range of IPS approaches in a single active period with the same energy budget, quantifying forward progress by the energy spent on effective execution. 
These fine-grained models are inefficient for processing long-term energy data, especially when iterative tests are needed for various system configurations. 

Besides models and simulators, hardware emulators of energy harvesters~\cite{10.1145/2668332.2668336, 10.1145/3356250.3360042} can provide repeatable power profiles recorded from energy harvesters for experimental comparisons. 
Though they provide practical results, hardware emulations are limited by hardware options and are generally impractical for performing long-term trials.

To address the above problem, we provide a reactive IPS model to estimate forward progress, as well as a simulation tool that enables fast exploration with long-term real-world environmental conditions. 
Further, we provide a sizing approach which recommends appropriate energy storage capacitance for deploying IPSs. 