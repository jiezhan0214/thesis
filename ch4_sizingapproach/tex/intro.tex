Having presented the effect of energy storage sizing, this chapter focusses on the energy storage sizing effect of the whole system under real-world energy conditions, and trading off multiple design factors. 
Current tools for ICSs (Section~\ref{section:review2}) are not practical for fast estimation of forward progress in a long-term deployment, and lack a method of sizing energy storage to improve forward progress while moderating the physical size and interruption periods. 
% the challenge of sizing energy storage to improve forward progress while moderating the physical size and interruption periods is largely unaddressed.
This chapter presents an approach for sizing energy storage in ICSs, quantifying and trading off forward progress, capacitor volume, and interruption periods. 
The main contributions are:
\begin{itemize}
    \item A model-based sizing approach that recommends appropriate energy storage capacitance in ICSs (Section~\ref{sec:c4_approach}).
    \item An evaluation of the impact of sizing in real-world conditions using real energy availability data (Section~\ref{sec:c4_demo}). 
    This includes a cost function-based method for trading off parameters. 
    In an example, this reduced capacitor volume and interruption periods by 83\% and 91\% respectively, while sacrificing 7\% of forward progress.
\end{itemize}

The associated simulation tool, coded in C, is available open-source~\footnote{\url{https://git.soton.ac.uk/energy-driven/energy-storage-sizing}}. 
