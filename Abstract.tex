\begin{abstract}
A growing number of autonomous sensor nodes are expected to be deployed in the Internet of Things. Energy harvesting has gained increasing attention for powering these nodes due to its features of long lifetime and energy independence. However, energy harvesting sources manifest intrinsic temporal and spatial variability, which conflicts with the requirement of stable power supply for conventional electronic designs. Energy-driven computing has recently developed to adapt system architecture to only ambient energy sources, with various specifications, such as prolonging active time, increasing energy efficiency, and ensuring forward execution, for different scenarios. While energy storage and energy harvester are critical components in terms of application performance and device dimensions, considerations on how to size energy storage and energy harvester have not been fully studied.

This thesis presents a study on the sizing issue of energy storage and energy harvester in deploying self-powered computing devices. A sizing method is proposed with respect to real-world energy conditions. A configurable system model is built to represent a typical execution process in energy harvesting computing. Based on this model, the sizing method is implemented. This sizing method suggests an efficient range of energy harvester sizes that balance average application performance and harvester dimensions. In this range of harvester sizes, results show that properly sizing energy storage leads to an execution speedup by 5.7-22.2\% under real-world deployment. Furthermore, exploration extends to the sizing effect of energy storage on a recent-published power adaptation method for storage-less energy-driven computing. 
\end{abstract}