\begin{abstract}
% Background: IoT and energy harvesting
Energy harvesting has become a promising power solution for the Internet of Things, liberating numerous wireless sensors from batteries and the power grid. 
Environmentally harvested power, however, is intrinsic variable and intermittent, which conflicts with conventional electronics. 
Conventionally, energy-harvesting devices buffer energy in large energy storage devices, such as rechargeable batteries or supercapacitors, to smooth out supply variability.
Unfortunately, these increase costs and device dimensions, raise pollution concerns, and have constrained lifespans.
% IPS
To work with only a small energy buffering capacitor, Intermittent-Powered Systems (IPSs) have been studied for intermittent supply. 
In IPSs, application forward progress, i.e. execution beneficial to the active application, is maintained by saving volatile computing state into non-volatile memory before power interruptions, and restored afterwards.

% Problem statement
While IPS research has focussed on efficiently sustaining computing state at the load side, system-wise energy budgeting in IPSs has not yet been widely studied.
% This thesis
This thesis investigates the energy budget of IPSs in order to improve forward progress. 
We studies the issues on sizing energy storage and setting voltage thresholds, both of which determine an energy budget. 
% 3 main contributions
The main contributions are:
(i) exploration and analysis of the energy storage sizing effect on IPS performance, where a reactive IPS model is proposed and validated to quantify and illustrate the relationship between energy storage capacitance and forward progress, showing up to 65\% forward progress improvement by sizing energy storage;
(ii) an energy storage sizing approach that recommends an appropriate energy storage size after analysing real-world energy availability data and trading off multiple design factors, achieving 93\% of the maximum forward progress while saving 83\% capacitor volume and 91\% interruption periods;
(iii) a runtime energy profiling and adaptation method for efficiently and reliably performing atomic tasks in cases of runtime-variable energy consumption, with results showing it can save manual profiling effort (within \SI{5}{\milli\volt} error), alleviates non-termination with even 68\% capacitance reduction, and improve forward progress by up to 98\% with runtime-variable workloads. 
% Concluding remark

\end{abstract}