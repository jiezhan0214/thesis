\section{Summary} \label{sec:c3_summary}

% Outline:
% Main work done, and results found
% Knowledge gained
% Link to the next chapter

While conventional IPSs have used minimal levels of capacitance, this chapter explored the energy storage sizing effect on IPS's forward progress.
A model of reactive IPSs was proposed to facilitate the understanding and design exploration of IPSs, explaining a mathematical relationship between forward progress and system parameters, e.g. energy storage capacitance and supply current.
We then configured the proposed model with an experimentally-profiled IPS platform, and utilised it to explore the energy storage sizing effect. 
The exploration results showed that sizing energy storage can improve forward progress by up to 65\%.
We also found that this improvement becomes significant when the supply current attenuates, which implies the importance of sizing energy storage in energy harvesting conditions where the supply current is low. 
The model was experimentally validated across a range of supply current and energy storage capacitance, showing a mean absolute percentage error of 0.5\%. 
We conclude that adding a relatively small amount of energy storage can significantly improve forward progress. 

The proposed model has demonstrated its potential for design exploration of IPSs. 
In the next chapter, we will incorporate the model in a simulation framework that recommends an appropriate energy storage size in real-world energy conditions. 
